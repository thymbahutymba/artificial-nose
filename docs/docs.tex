\documentclass{article}
\usepackage[utf8]{inputenc}

\title{Artificial Nose}
\author{Luca Belluardo e Andrea Stevanato}
\date{\today}

\begin{document}

\maketitle

\section{Introduction}
In this project a real-time application is developed to recognize smells from an artificial nose. This sensor is an air quality gas sensor.

We have 7 periodic tasks, a graphic thread (GT), a sensor thread (ST), a neural network thread (NNT), a keyboard thread (KT) and a store image thread (SIT). The main function sets everything up for the tasks, except (SIT), to execute, then creating them and waiting for the termination of the (KT). The (SIT) is started from the (KT) when the user writes a directory name and presses the ENTER key, and is is killed from the (KT) when the ENTER key is pressed again.The sensor runs on Arduino, where a thread reads from the sensor. All the tasks terminate when the user presses the ESC key on the keyboard.

The rest of the article is so structured: in section 2 the tasks are explained one at a time, in the section 3...

\section{The tasks}
In the following subsections all the tasks and their functions are explained in detail.

\subsection{Main function}
In the main function all tasks and mutexes are initialized. Then it creates the tasks and starts them. The mutexes are two, one for the data readed from the sensor and the other for the results of the neural network. Then the main proceeds to start up allegro and it waits for ESC key to be pressed. When the ESC key is pressed, the main function cancels all the threads and it closes allegro.

The main structure is the following: 

                         (inserire codice main)
\subsection{Graphic Task}

\end{document}
